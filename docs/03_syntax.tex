%
% Copyright 2022 Krister Joas
%
% Licensed under the Apache License, Version 2.0 (the "License");
% you may not use this file except in compliance with the License.
% You may obtain a copy of the License at
%
%     http://www.apache.org/licenses/LICENSE-2.0
%
% Unless required by applicable law or agreed to in writing, software
% distributed under the License is distributed on an "AS IS" BASIS,
% WITHOUT WARRANTIES OR CONDITIONS OF ANY KIND, either express or implied.
% See the License for the specific language governing permissions and
% limitations under the License.
%
\section{Lips Input Syntax}
The types of objects recognized by the lisp reader are integers,
floats, literal atoms, lists, and strings.  All objects except lists
and dotted pairs are called atoms.

Strings start and end with a double quote, `\lisp{"}'.  To enter a
double quote in a string preceed it by a backslash, `\lisp{\char`\\}'.
For instance, \lisp{"Foo bar"} is a string and so is \lisp{"I like
  \char`\\"lips\char`\\""} which contains two double quotes.  Newlines
may be embedded inside strings without escaping them.

Lists start with a left parenthesis followed by zero or more lisp
expressions separated by separators (see below).  The list ends with a
matching right parenthesis.  Separators breaks the atoms, and by
default these are blanks, tabs, and newlines.  There is also a class
of characters called break characters.  These characters breaks
literal atoms and there is no need to add separators around them.
Break characters are also literal atoms.  The default break characters
are: `\lisp{(}', `\lisp{)}', `\lisp{\&}', `\lisp{<}', `\lisp{>}' and
`\lisp{|}'.  Thus,

\begin{quote}
  \lisp{foo\\
    (foo)\\
    (foo fie fum)\\
    (foo (fie (x y (z))) (fum))}
\end{quote}

are all valid lisp expressions.  The nesting of parenthesis may have
arbitrary depth.

\lips\ supports super parenthesis. Super parenthesis are square
brackets and when an opening square bracket is closed by a matching
closing square bracket any missing round parenthesis is added. A final
closing square bracket closes any open round parenthesis and finishes
the expression. Here are some examples:

\begin{examples}
  \ex{(cond [(null nil) "hello"] (t "world"))}{"hello"}
  \ex{(cons 'a (cons 'b (cons 'c]}{(a b c)}
\end{examples}

The elements of lists are stored in the car part of the cons cells,
and are then linked together by the cdr's.  The last cdr is always
\NIL.  It's possible to enter so called dotted pairs in \lips.  A
dotted pair is a cons cell with two expressions with no restrictions
in the car and cdr of the cell.  Dotted pairs are entered starting
with a left parenthesis, a \lips\ expression, followed by a dot, `.',
another expression, and terminated with a right parenthesis.
\lisp{(a\ .\ b)} is an example of a dotted pair. Note that the blanks
around the dot are necessary, the dot is not a break character.  The
dot is recognized in this special way only if it occurs as the second
element from the end of a list.  In other cases it is treated as an
ordinary atom.  The list \lisp{(.\ .\ .\ .)} is a list with three
elements ending in a dotted pair.

A list is just a special case of dotted pairs.  \lisp{(a .\ (b
  .\ nil))} is equivalent to \lisp{(a b)}.  The second format is just
a convenience since lists are so common.

Integers consist of a sequence of digits.  No check for overflow is
made.  Floats are a bit complicated, but in most cases an atom that
looks like float is a float.  A float is a sequence of digits that
have either a decimal point or a exponentiation character, `e',
inside.  At most one decimal point is allowed.  If the exponentiation
character is given it must be followed by at least one digit.  If
these rules are not followed the atom will be interpreted as a literal
atom.

Examples:
\begin{tabbing}
\lisp{foo}\hspace*{1cm}\=literal atom\\
\lisp{"fie"}      \>string\\
\lisp{123}        \>integer\\
\lisp{12a}        \>literal atom\\
\lisp{1.0}        \>float\\
\lisp{1e5}        \>float\\
\lisp{1e}         \>literal atom\\
\lisp{.e4}        \>literal atom\\
\lisp{1.4.}       \>literal atom
\end{tabbing}

All expressions typed at the top level prompt are treated as lists.
This means that \lips\ supplies an extra pair of matching parenthesis
around all expressions.  If the first expression of a line is an atom,
and not a list, input terminates with either a return (providing that
parenthesis in subexpressions match), or an extra right parenthesis.
In the first case, a matching pair of parenthesis are added
surrounding the line, in the second case an extra left parenthesis is
added as the first character.  Again, if a left parenthesis is missing
a matching parenthesis is inserted.

Typing the outermost parenthesis explicitly will make \lips\ print out
the return value of the expression.  This is the way normal lisp
systems behave, but when used only as a command shell the return value
is most often uninteresting.  The return value is also stored in the
variable $value$.

Comments are allowed in \lips\ files.  In order to allow for the
\textsc{Unix} ``shebang'' interpreter directive a `\lisp{\#}'
character in the first column of a line is recognized as a comment
that ends with a newline. A \lisp{\#} in any other colmn is treated as
a regular character.  Comments may also start with the semicolon
character and the comment again continues until the end of the line.
