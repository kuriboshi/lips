%
% Copyright 2022 Krister Joas
%
% Licensed under the Apache License, Version 2.0 (the "License");
% you may not use this file except in compliance with the License.
% You may obtain a copy of the License at
%
%     http://www.apache.org/licenses/LICENSE-2.0
%
% Unless required by applicable law or agreed to in writing, software
% distributed under the License is distributed on an "AS IS" BASIS,
% WITHOUT WARRANTIES OR CONDITIONS OF ANY KIND, either express or implied.
% See the License for the specific language governing permissions and
% limitations under the License.
%
\section{The Evaluation Process}
The \lips\ interpreter has ``dynamic scope'' and ``shallow'' binding.
``Dynamic scope'' means that free variables are looked up in the call
stack. For example:

\example{((lambda (a) ((lambda (b) (plus a b)) 1)) 2)}{3}

``Lexical scope'' can be simulated using the \lisp{closure} function
(see section \ref{closure} on page \pageref{closure}) which creates a
lexical scope for variables.

``Shallow'' binding means that every time a symbol is bound to a value
the previous value is pushed onto a stack.  The current value of an
atom is stored in the ``value cell'' and can be retrieved immediately.
The opposite of ``shallow'' binding, ``deep'' binding, in contrast has
to traverse the stack to find the current value of an atom.

In \lips\ there is no top level value.  There is also no ``function
definition cell'', only a ``value cell''.  Defining a function and
binding it to a symbol simply sets the value cell for that symbol.
Symbols are case sensitive and all system symbols are in lower case.
\lips\ does not use a stop and sweep garbage collector.  Instead
objects are reference counted and freed when the reference count goes
down to zero.

On the top level \lips\ reads expressions from standard in and
evaluates the expressions.  Commands entered on the top level are
always treated as functions, even if no parameters are given or if the
expression is a single atom.  In order to print the value of a
variable the function print must be used.

If the expression to be evaluated is a list the first element (the
car) in the list is evaluated and then applied to the arguments.  The
car of the expression is reevalutated until either a proper functional
form is recognized or if it's evaluated to an illegal functional form,
in which case an error is signalled.

If the functional form is an unbound atom, \lips\ looks for the
property \lips{autoload} on the property list of the atom.  If it is
found, and the value is a symbol or a string, the file with that name
is loaded (if possible).  The interpreter then checks if the atom is
no longer unbound, in wich case evaluation continues.  If the atom
doesn't have the \lips{autoload} property, or loading the file didn't
define the symbol, \lips\ looks under the property \lisp{alias}.  The
atom is replaced with the expression stored under that property, if
any, and the rest of the expression is appended at the end and the
evaluation process continues.  When all else fails it is assumed the
atom stands for an executable command.  The return value of an
executable command is, at present, always \T.

Note that the arguments for the command are never evaluated.  If you
want them evaluated you must use the function \lisp{apply}.
