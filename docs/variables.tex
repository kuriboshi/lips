%
% Copyright 2022 Krister Joas
%
% Licensed under the Apache License, Version 2.0 (the "License");
% you may not use this file except in compliance with the License.
% You may obtain a copy of the License at
%
%     http://www.apache.org/licenses/LICENSE-2.0
%
% Unless required by applicable law or agreed to in writing, software
% distributed under the License is distributed on an "AS IS" BASIS,
% WITHOUT WARRANTIES OR CONDITIONS OF ANY KIND, either express or implied.
% See the License for the specific language governing permissions and
% limitations under the License.
%
\section{Variables}
What follows is a list of all user accessable variables that in some
way guides the behavior of \lips.
\begin{description}
\item[histmax] Controls the number of commands to save on the history
  list.
\item[verboseflg] If verboseflg is non-\NIL, \lips\ prints some
  messages whenever a function is redefined or a variable is reset to
  a new value.
\item[path] The \lisp{path} variable contains a list of directories to
  be searched for executable programs.
\item[home] This variable contains the home directory of the user.
\item[history] In the \lisp{history} variable the history list is
  built.  This is for internal use and proper functioning of \lips\ is
  not garantueed if history is set to funny values.
\item[histnum] The variable \lisp{histnum} contains the number on the
  current \lips\ interaction.  The same warnings apply to histnum as
  to history.
\item[prompt] This variable contains the string that is printed as the
  prompt.  An exclamation mark is replaced with the current history
  number from \lisp{histnum}.  The default prompt is ``!\_''.
\item[brkprompt] Same as \lisp{prompt} but controls prompting in a break.
  The default is ``!:''.
\item[promptform] This variable is evaluated before the prompt is
  printed.  If an error occurs during evaluation of promptform it is
  reset to the default prompt.
\end{description}
