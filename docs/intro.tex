\section{Introduction}
\lips\ is a embeddable lisp interpreter written in \textsf{C++-20}.
In addition to a set of typical lisp functions, \lips\ offers
functionality to make it work as a command shell.  It includes
functions for program composition with pipes and input/output
redirection.  \lips\ makes use of read macros and transformation of
the input to make the input syntax resemble the standard shells.

\lips\ can be used as a lisp interpreter which can be linked with
other applications.  The \lips\ shell is an example of such an
application.

\lips\ is inspired by \textsc{Interlisp} and functions are usually
named as they are named in \textsc{Interlisp}.  Functions tend to
behave in the same way as they do in \textsc{Interlisp}.  There are
exceptions and \lips\ is not a faithful implementation of
\textsc{Interlisp} by any stretch.

This document is a reference manual for \lips\ and should be adequate
for getting started.  See section \ref{functions} page
\pageref{functions} for a complete list of lisp functions provided.

When \lips\ is started it reads and evaluates expressions in a central
initialization file.  This file is called \texttt{lipsrc}, and is
located in some public directory, typically
\texttt{/usr/local/share/lips}.  After reading this file \lips\ looks
for the file \texttt{.lipsrc} in the users home directory.  If it is
found, the expressions in it are read and evaluated. \lips\ then reads
expressions from any file given on the command line.

When the startup process is done, \lips\ prompts with the current
history number followed by an underscore.  All commands are saved on a
history list, and may be recalled at any time.  However, a maximum
number of events saved on the history list is limited to the value of
the variable \lisp{histmax}.  This variable may be set by the user to
any value.
